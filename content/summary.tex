%----------------------------------------------------------------------------
\chapter{Összefoglalás}
%----------------------------------------------------------------------------

A célként kitűzött szcenárió alapú monitor generátor kibővítése sikerült.

A szöveges szcenárió leíró nyelv támogatja \textit{TPSC} diagramok specifikálását.
Az automata generátor támogatja a \textit{TPSC} tulajdonságokhoz tartozó minta automaták generálását és képes az üzenet paramétereket is értelmezni.
A generátor támogatja az \textit{alt}, \textit{loop} és \textit{par} operátorokat tartalmazó \textit{TPSC}-khez tartozó időzített automaták generálását.

A szcenáriókat a szöveges leírásuk alapján diagramok formájában vizualizálom.
Egy \textit{XML} generátor teszi ezt lehetővé, amely a szöveges leírás alapján elkészíti a hozzá tartozó diagram \textit{XML} leírását.

A monitor forráskód generátor a szcenárióhoz tartozó automata alapján képes egy monitor forráskódjának generálására.
Legenerálja a megfelelő interfészeket amik a monitor és rendszer közti kommunikációhoz szükségesek.
Ha az üzenetek megfigyeléséhez szükséges segédfüggvényeket a kommunikációs infrastruktúrához megvalósítják, akkor a monitor képes a rendszer viselkedésének ellenőrzésére.
A monitor forráskód generátor a \textit{par}, \textit{loop} és \textit{alt} operátorokat támogatja.
Továbbá az időzítési feltételeket tartalmazó üzeneteket is tudja értelmezni.

A generált monitor forráskód helyességét alapos tesztelés segítségével ellenőriztem.
A monitor forráskód szisztematikus helyességét egy \textit{CI} rendszerrel ellenőrzöm, amely minden változtatás esetén ellenőrzi, hogy a projekt tesztjei helyesek e.

Végezetül a monitor komponenst illesztettem a \textit{Gamma} keretrendszerhez, így demonstrálva a monitorozás működését egy elosztott komponensű rendszer viselkedésének ellenőrzésével.

A monitor forráskód generátor tovább bővíthető úgy hogy, láncolt megkötéseket is támogasson.
A szcenárió követelményünkben hasznos lehet olyan megkötéseket definiálni amelyek nem egy üzenet halmazra vonatkoznak hanem egy kisebb üzenet szekvenciára.
Elképzelhető olyan eset ahol adott üzenetek külön-külön nem jelentenek hibát a rendszer működésére nézve, viszont ha adott sorrendben érkeznek az már hibás lehet.
Ehhez bevezethető egy új nyelvi elem a meglévő \textit{TPSC} leíró nyelvbe, amely ilyen láncolt megkötések megadására szolgál.
Az automata generátorhoz is hozzáadhatok az új nyelvi elemhez tartozó automata minták.