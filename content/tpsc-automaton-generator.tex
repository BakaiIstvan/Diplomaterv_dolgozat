%----------------------------------------------------------------------------
\chapter{Időzített automata generátor}
%----------------------------------------------------------------------------

%----------------------------------------------------------------------------
\section{Az automata generátor célja}
%----------------------------------------------------------------------------

A diplomatervezés során elkészített automata generátort kibővítettem úgy, hogy támogassa a TPSC elemekhez tartozó automata minták generálását. Bemenetként egy TPSC scenario szöveges leírását kapja meg amiből a minta alapú módszerrel generál egy TA automatát.

A 13. ábrán látható, hogy a monitor generátor támogatja az alt, par és loop operátorokat tartalmazó TPSC-khez tartozó TA-k generálását is. Továbbá a generátor képes az üzenet paraméterek kezelésére. Például az alt operátor feltételét képes feldolgozni és azt elhelyezni a generált automata megfelelő élén.

%----------------------------------------------------------------------------
\section{Az automata generátor megvalósítása}
%----------------------------------------------------------------------------
A generátorhoz az Xtend technológiát használtam. Minden egyes TPSC üzenethez legenerálja a hozzá tartozó minta automatát, majd elvégzi azok összecsatolását.

Az időzített automaták generálásához egy adatstruktúrát definiáltam, amely a következő Java osztályokból áll:
\begin{itemize}
    \item State
    \item StateType
    \item Transition
    \item Automaton
    \item Specification
\end{itemize}

A 14. ábrán látható az adatstruktúra UML osztály diagramja.
% TODO: kép beíllesztése

Az automatában lévő állapotok implementációja a State osztályban található. Két attribútuma van: id(String), a címkéje tárolására, és type(StateType), az állapot típusa.

Az állapot típusának a megadására a StateType enum osztályt definiáltam. NORMAL, ACCEPT, FINAL értékeket lehet benne eltárolni. Az átmenetek implementációjáért felelős osztály a Transition. Három tag változója van: id(String) az üzenet, sender(State), a feladó állapot, és receiver(State) a fogadó állapot.

Az időzített automata implementációja az Automaton osztályban található. Itt tároljuk az automatában lévő állapotokat és a köztük lévő átmeneteket egy-egy listában. Az Automaton osztály addState(State) és addTransition(Transition) függvényeivel lehet új állapotot és átmenetet hozzáadni az automatához, a collapse(Automaton) függvényével pedig két automatát egyesíteni. Ezt a függvényt használtam az implementációban a minta automaták egyesítésére. Ezen kívül az osztálynak van egy merge(ArrayList<Automaton>) függvénye. Ez az előző fejezetben definiált merge függvény implementációja.

A Specification osztály feladata, hogy összeállítsa a szöveges leírásban specifikált TPSC scenariohoz tartozó időzített automatát. Ezt követően az automata Never Claim leírását egy .txt kiterjesztésű fájlba írja.

%----------------------------------------------------------------------------
\section{Minta példa}
%----------------------------------------------------------------------------

% TODO: képek beíllesztése
A fenti 14., 15. és 16. ábrákon látható, hogy a generátor milyen időzített automatát generál a megadott TPSC scenario-ból.