%----------------------------------------------------------------------------
\chapter{\bevezetes}
%----------------------------------------------------------------------------
Diplomamunkám célja, hogy a „\textit{Monitor komponensek generálása kontextusfüggő viselkedés ellenőrzése}” \cite{Bakai} című \textit{BSc} szakdolgozatom során elkészített monitor komponens generátort kibővítsem úgy, hogy támogassa időzítési feltételek megadását.
Szakdolgozatomban bemutatott monitor komponens generátor képes \textit{PSC} (\textit{Property Sequence Chart}) \cite{PSC1} diagramok szöveges leírásából \textit{Büchi} \cite{PSC1} automatákat generálni.
A generátor kimenete a generált \textit{Büchi} automata \textit{Java} implementációja, amely felhasználható monitorozásra.
Az \textit{Önálló laboratórium} keretében a szakdolgozatban definiált szöveges \textit{PSC} diagram leíró nyelv kibővítése volt a feladatom úgy, hogy időzítési feltételeket is meg lehessen adni a szcenáriókban.
Az időzített feltételek bevezetéséhez a \textit{Timed Property Sequence Chart} \cite{TPSC1} formalizmust választottam.
A meglévő \textit{PSC} szöveges leíróhoz új nyelvi elemeket definiáltam, amelyekkel a \textit{TPSC} elemeket szöveges formában lehet leírni.
Ezt követően az automata generátort kellett úgy kibővíteni, hogy a \textit{TPSC} diagramokból tudjon \textit{TA} időzített automatákat \cite{TPSC1} generálni.
Ennek érdekében, a szakdolgozatom során készített \textit{Büchi} automata generátort bővítettem ki úgy, hogy az új \textit{TPSC} szöveges leírásokból időzített automatákat generáljon.
Egy monitor forráskód generátor pedig az automata alapján elkészítheti a monitor forráskódját.

A szöveges \textit{TPSC} szcenárió leírása alapján el kell készítenünk a diagram vizualizációját, hogy grafikusan megtekinthessük a definiált szcenáriót.
Ehhez felhasználható a "\textit{Modell alapú rendszertervezés}" című tárgy keretében általam készített \textit{PSC} diagramszerkesztő alkalmazás.
A következő a generált monitor forráskód tesztelése, majd ezt a \textit{Gamma} keretrendszerhez \cite{Gamma} kell illesztenünk.
Ezzel az a célom, hogy elosztott komponens alapú rendszerek szimulációja közben monitorozható legyen a \textit{TPSC} üzenet szekvencia specifikáció teljesülése, illetve az ebben rögzített tulajdonságok megsértése.

A hátramaradó feladatok a monitor forráskód generátor kibővítése és annak tesztelése, a diagramok vizualizációja és a monitor komponens illesztése a \textit{Gamma} keretrendszerhez.

A dolgozatomat a háttérismeretek összefoglaló fejezettel kezdem.
Először bemutatom a legelterjedtebb formalizmusokat, amelyek időfüggő viselkedés specifikálására szolgálnak.
Ezt követi a \textit{TPSC} formalizmus és a felhasznált technológiák bemutatása.
A dolgozatomat a harmadik fejezetben a kibővített szöveges \textit{TPSC} leíró nyelv bemutatásával folytatom.
Ezt a \textit{TPSC} specifikációk vizualizációjáról szóló fejezet követi.

Az ötödik fejezet témája a monitor forráskód generálás és annak tesztelése.
Ezt követi a hatodik fejezet, amely bemutatom a monitor komponens a \textit{Gamma} keretrendszerhez való illesztését mutatja be, valamint az elosztott komponensű rendszerek monitorozását.
A \textit{Gamma} keretrendszer komponens alapú elosztott rendszerek tervezését, kódgenerálását segíti, így érdemes a keretrendszerhez a generált monitort illeszteni.
A dolgozatomat egy összefoglalóval zárom.\footnote{A diplomatervhez készített alkalmazások forráskódjai a következő \textit{Github} \textit{repository}-ban elérhetők:
\url{https://github.com/BakaiIstvan/Minotor}.}