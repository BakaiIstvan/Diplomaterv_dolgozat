%----------------------------------------------------------------------------
\chapter{\bevezetes}
%----------------------------------------------------------------------------
Diplomamunkám során az volt a cél, hogy a „\textit{Monitor komponensek generálása kontextusfüggő viselkedés ellenőrzése}” \cite{Bakai} című szakdolgozatom során elkészített monitor komponens generátort kibővítsem úgy, hogy támogassa időzítési feltételek megadását.
A szakdolgozat során egy olyan monitor komponens generátor készült el, amely képes volt \textit{PSC} (\textit{Property Sequence Chart}) \cite{PSC1} diagramok szöveges leírásából \textit{Büchi} \cite{PSC1} automatákat generálni.
A generátornak a kimenete a generált \textit{Büchi} automata \textit{Java} implementációja.
A generált \textit{Büchi} automatát lehet felhasználni monitorozásra.
Az \textit{Önálló laboratórium} keretében az volt a feladatom, hogy a szakdolgozatom során definiált szöveges \textit{PSC} diagram leíró nyelvet kibővítsem úgy, hogy időzítési feltételeket is meg lehessen adni a szcenáriókban.
A \textit{Timed Property Sequence Chart} \cite{TPSC1} formalizmust választottam az időzített feltételek bevezetéséhez.
Definiáltam a meglévő \textit{PSC} szöveges leíróhoz új nyelvi elemeket, amelyekkel a \textit{TPSC} elemeket lehet szöveges formában leírni.
Ezután az automata generátort kellett úgy kibővíteni, hogy a \textit{TPSC} diagramokból tudjon \textit{TA} időzített automatákat \cite{TPSC1} generálni.
Ennek érdekében, a szakdolgozat során készített \textit{Büchi} automata generátort bővítettem ki úgy, hogy az új \textit{TPSC} szöveges leírásokból időzített automatákat generáljon.
Egy monitor forráskód generátor pedig az automata alapján elkészítheti a monitor forráskódját.

A szöveges \textit{TPSC} szcenárió leírása alapján el kell készítenünk a diagram vizualizációját, hogy grafikusan megtekinthessük a definiált szcenáriót.
Ehhez felhasználható a "\textit{Modell alapú rendszertervezés}" tárgy keretében készített \textit{PSC} diagram szerkesztő alkalmazás.
A következő a generált monitor forráskód tesztelése, majd ezután ezt illesszük a \textit{Gamma} keretrendszerhez \cite{Gamma}.
Ezzel az a célunk, hogy elosztott komponens alapú rendszerek szimulációja közben monitorozható legyen a \textit{TPSC} üzenet szekvencia specifikáció teljesülése illetve az ebben rögzített tulajdonságok megsértése.

Elkészítettem a monitor forráskód generátort és elkezdtem annak tesztelését.
A hátramaradó feladatok a tesztelés befejezése, a diagramok vizualizációja és a monitor komponens illesztése a \textit{Gamma} keretrendszerhez.

A dolgozatomat a háttérismeretek összefoglaló fejezettel kezdem.
Először bemutatom a legelterjedtebb formalizmusokat, amelyek időfüggő viselkedés specifikálására szolgálnak.
Ezután a \textit{TPSC} formalizmust mutatom be és a felhasznált technológiákat.
A dolgozatomat a kibővített szöveges \textit{TPSC} leíró nyelv bemutatásával folytatom, amit a harmadik fejezetben írok le.
Ezt a \textit{TPSC} specifikációk vizualizációjáról szóló fejezet követi.

Az ötödik fejezet a monitor forráskód generálásról szól és annak teszteléséről.
Ezt követi a hatodik fejezet, amely a monitor komponens illesztését mutatja be a \textit{Gamma} keretrendszerhez és elosztott komponensű rendszerek monitorozását.
Ez a keretrendszer komponens alapú elosztott rendszerek tervezését, kódgenerálását segíti, így ehhez érdemes a monitor generálást illeszteni.
A dolgozatomat egy összefoglalóval zárom.

A diplomatervhez készített alkalmazások forráskódjai a következő \textit{Github} \textit{repository}-ban elérhetők:
\url{https://github.com/BakaiIstvan/Minotor}.