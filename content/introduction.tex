%----------------------------------------------------------------------------
\chapter{\bevezetes}
%----------------------------------------------------------------------------
Diplomamunkám során az volt a cél, hogy a „\textit{Monitor komponensek generálása kontextusfüggő viselkedés ellenőrzése}” című szakdolgozatom során elkészített monitor komponens generátort kibővítsem úgy, hogy támogassa időzítési feltételek megadását.
Az \textit{Önálló laboratórium} keretében az volt a feladatom, hogy a szakdolgozatom során definiált szöveges \textit{PSC} diagram leíró nyelvet kibővitsem a \textit{TPSC} elemeivel.
Ezután az automata generátort kell úgy kibővíteni, hogy a \textit{TPSC} diagramokból tudjon \textit{TA} automatákat generálni.
Egy monitor forráskód generátor pedig az automata alapján elkészítheti a monitor forráskódját.

A szöveges \textit{TPSC} szcenárió leírásához el kell készítenünk a diagram vizualizációját, hogy grafikusan megtekinthessük a definiált szcenáriót.
Ehhez felhasználható a "\textit{Modell alapú rendszertervezés}" tárgy keretében készített \textit{PSC} diagram szerkesztő alkalmazás.
A következő a generált monitor forráskód tesztelése, majd ezután ezt illeszük a \textit{Gamma} keretrendszerhez.
Ezzel az a célunk, hogy elosztott komponens alapú rendszerek szimulációja közben monitorozható legyen a \textit{TPSC} üzenet szekvencia specifikáció teljesülése illetve az ebben rögzített tulajdonságok megsértése.

A \textit{Diplomatervezés 1} tárgy keretében elkészítettem a monitor forráskód generátort és elkezdtem annak tesztélését.
A hátramaradó feladatok közé tartozik a tesztelés befejezése, a diagramok vizualizációja és a monitor komponens illesztése a \textit{Gamma} keretrendszerhez.

A dolgozatomat a háttérismeretek összefoglalásával kezdem.
Előszőr bemutatom a legelterjetebb formalizmusokat, amelyek időfüggő viselkedés specifikálására szolgálnak.
Ezután a \textit{TPSC} formalizmust mutatom be és a felhasznált technológiákat.
A dolgozatomat a kivőbített szöveges \textit{TPSC} leíró nyelv bemutatásával folytatom.
Ezt a \textit{TPSC} specifikációk vizualizációjáról szoló fejezet követi.

A következő fejezet a monitor forráskód generálásról szól és annak teszteléséről.
Ezt követi egy fejezet, amely a monitor komponens illesztését mutatja be a \textit{Gamma} keretrendszerhez és elosztott komponensű rendszerek monitorozását.
A dolgozatomat egy összefoglalóval zárom.