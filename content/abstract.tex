\pagenumbering{roman}
\setcounter{page}{1}

\selecthungarian

%----------------------------------------------------------------------------
% Abstract in Hungarian
%----------------------------------------------------------------------------
\chapter*{Kivonat}\addcontentsline{toc}{chapter}{Kivonat}

A monitorozással történő hibadetektálás létfontosságú egy rendszer működtetésében és karbantartásában. A monitorozás sok hibát fel tud deríteni, amiket a tesztek nem feltétlenül tudnak kideríteni.

Az önálló laboratórium feladat célja az volt, hogy a „Monitor komponensek generálása kontextusfüggő viselkedés ellenőrzésére” című szakdolgozatomban definiált szöveges PSC (Property Sequence Chart) leíró nyelvet kibővítsem úgy, hogy időzítési feltételeket is meg lehessen adni a követelményben. Ilyen követelményeket egyszerűen specifikálhatunk TPSC (Timed Property Sequence Chart) diagramokkal. Az Xtext alapú nyelvet kiegészítettem a TPSC tulajdonságaival.

A monitor generálás következő lépése, hogy a PSC diagramokból TA időzített automatákat generálunk (Timed Automaton). A TA fogja megadni, hogy a megfigyelt kommunikáció helyes viselkedést jelent-e. A szakdolgozatomban készített automata generátort kibővítettem és most már képes a minta alapú módszert használva TA automatákat generálni a TPSC diagramjainkból. A generátor előállítja az automatát.

A szöveges scenario leírásból generált automata alapján legenerálható a monitor forráskódja. A monitor forráskód generátor előállítja a monitor Java implementációját, ami képes egy rendszer monitorozására adott követelmény alapján.



\vfill
\selectenglish


%----------------------------------------------------------------------------
% Abstract in English
%----------------------------------------------------------------------------
\chapter*{Abstract}\addcontentsline{toc}{chapter}{Abstract}

% TODO: összefoglaló angol verziójának megírása
This document is a \LaTeX-based skeleton for BSc/MSc~theses of students at the Electrical Engineering and Informatics Faculty, Budapest University of Technology and Economics. The usage of this skeleton is optional. It has been tested with the \emph{TeXLive} \TeX~implementation, and it requires the PDF-\LaTeX~compiler.


\vfill
\selectthesislanguage

\newcounter{romanPage}
\setcounter{romanPage}{\value{page}}
\stepcounter{romanPage}