\pagenumbering{roman}
\setcounter{page}{1}

\selecthungarian

%----------------------------------------------------------------------------
% Abstract in Hungarian
%----------------------------------------------------------------------------
\chapter*{Kivonat}\addcontentsline{toc}{chapter}{Kivonat}

A futásidőbeli monitorozással történő hibadetektálás kiemelt fontosságú egy kritikus rendszer működtetésében és karbantartásában.
A monitorozás sok hibát fel tud deríteni, amiket a tesztek nem feltétlenül tudnak.

A diplomaterv célja az volt, hogy a korábban elkészített monitor komponens generátort kiegészítsem úgy, hogy időzített üzenet szekvencia specifikáció alapján is képes legyen monitor komponenseket generálni.
A generált monitor feladata az üzenet szekvencia által specifikált viselkedés ellenőrzése.
Ilyen követelményeket egyszerűen specifikálhatunk \textit{TPSC} (\textit{Timed Property Sequence Chart}) diagramokkal.
A szakdolgozatom során elkészített \textit{Xtext} alapú \textit{PSC} nyelvet kiegészítettem a \textit{TPSC} tulajdonságaival.

A monitor generálás következő lépése, hogy a \textit{TPSC} diagramokból időzített automatákat generálunk (\textit{Timed Automaton}).
A \textit{TA} fogja megadni, hogy a megfigyelt kommunikáció helyes viselkedést jelent-e.
A szakdolgozatomban készített automata generátort kibővítettem úgy, hogy képes legyen a minta alapú módszert használva \textit{TA} automatákat generálni a \textit{TPSC} diagramokból.

A szöveges szcenárió leírásból generált automata alapján legenerálható a monitor forráskódja.
A monitor forráskód generátor előállítja a monitor \textit{Java} implementációját, ami képes egy rendszer monitorozására adott követelmény alapján.

A diplomaterv további feladatai közzé tartozik a \textit{TPSC} szcenáriók vizualizációja, a generált forráskódok szisztematikus tesztelése és a generált monitor komponens illesztése a \textit{Gamma} keretrendszerrel.
A cél az, hogy elosztott komponens alapú rendszerek szimulációja közben monitorozható legyen a \textit{TPSC} üzenet szekvencia specifikáció teljesülése illetve az ebben rögzített tulajdonságok megsértése.
Végezetül az utolsó feladat a monitorozás működésének demonstrációja.

\vfill
\selectenglish


%----------------------------------------------------------------------------
% Abstract in English
%----------------------------------------------------------------------------
\chapter*{Abstract}\addcontentsline{toc}{chapter}{Abstract}

Runtime verification of a critical system is essential for its operation and maintenance.
With runtime verification we can discover a lot of errors that may stay undiscovered after testing.

The goal of this thesis is to further enhance the previously created monitor generator, so that it is able to generate monitor source code from scenario containing clock constraints.
We can specify this sort of scenario using the \textit{TPSC} (\textit{Timed Property Sequence Chart}) diagrams.
During my \textit{BSc} thesis, I have developped a language using \textit{Xtext} for specifying \textit{PSC} diagrams via text.
I have extended this language so that \textit{TPSC} diagrams can be created using a textual format.

The next step of monitor generation is to convert the \textit{TPSC} scenarios into \textit{Timed Automata}.
The \textit{TA} will serve as the representation which indicates if the system's behaviour has satisfied the TPSC requirement or not.
I have further enhanced the automaton generator created during my \textit{BSc} thesis so that it is able to generate \textit{Timed Automata} from \textit{TPSC}s.

We can synthesize the monitor source code using the generated automaton from the \textit{TPSC} scenario.
The monitor source code generator creates the \textit{Java} implementation of the monitor which is able to perform runtime verification of a system based on a specified requirement.

The remaining tasks for the thesis are the visualization of \textit{TPSC} scenarios, the systematic testing of generated monitor source code and the integration of the generated monitor with the \textit{Gamma} framework.
The goal is to be able to monitor component-based system based on a requirement specified with a \textit{TPSC} scenario.
The last task is to demonstrate the runtime verification of a system.

\vfill
\selectthesislanguage

\newcounter{romanPage}
\setcounter{romanPage}{\value{page}}
\stepcounter{romanPage}