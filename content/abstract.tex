\pagenumbering{roman}
\setcounter{page}{1}

\selecthungarian

%----------------------------------------------------------------------------
% Abstract in Hungarian
%----------------------------------------------------------------------------
\chapter*{Kivonat}\addcontentsline{toc}{chapter}{Kivonat}

A monitorozással történő hibadetektálás kiemelt fontosságú egy kritikus rendszer működtetésében és karbantartásában.
A monitorozás sok hibát fel tud deríteni, amiket a tesztek nem feltétlenül tudnak.

A diplomaterv célja az volt, hogy a korábban elkészített monitor komponens generátort kiegészítsem úgy, hogy időzített üzenet szekvencia specifikáció alapján is képes legyen monitor komponenseket generálni.
Ilyen követelményeket egyszerűen specifikálhatunk \textit{TPSC} (Timed Property Sequence Chart) diagramokkal.
A szakdolgozatom során elkészített \textit{Xtext} alapú \textit{PSC} nyelvet kiegészítettem a \textit{TPSC} tulajdonságaival.

A monitor generálás következő lépése, hogy a \textit{TPSC} diagramokból időzített automatákat generálunk (Timed Automaton).
A \textit{TA} fogja megadni, hogy a megfigyelt kommunikáció helyes viselkedést jelent-e.
A szakdolgozatomban készített automata generátort kibővítettem úgy, hogy képes legyen a minta alapú módszert használva \textit{TA} automatákat generálni a \textit{TPSC} diagramokból.

A szöveges scenario leírásból generált automata alapján legenerálható a monitor forráskódja.
A monitor forráskód generátor előállítja a monitor \textit{Java} implementációját, ami képes egy rendszer monitorozására adott követelmény alapján.

A diplomaterv további feladatai közzé tartozik a \textit{TPSC} scenario-k vizualizációja, a generált forráskódok folyamatos tesztelése és a generált monitor komponens illesztése a \textit{Gamma} keretrendszerrel.
A cél az, hogy elosztott komponens alapú rendszerek szimulációja közben monitorozható legyen a \textit{TPSC} üzenet szekvencia specifikáció teljesülése illetve az ebben rögzített tulajdonságok megsértése.
Továbbá a monitorozás működését is demonstrálni kell.

\vfill
\selectenglish


%----------------------------------------------------------------------------
% Abstract in English
%----------------------------------------------------------------------------
\chapter*{Abstract}\addcontentsline{toc}{chapter}{Abstract}



\vfill
\selectthesislanguage

\newcounter{romanPage}
\setcounter{romanPage}{\value{page}}
\stepcounter{romanPage}