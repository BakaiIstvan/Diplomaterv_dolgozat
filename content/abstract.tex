\pagenumbering{roman}
\setcounter{page}{1}

\selecthungarian

%----------------------------------------------------------------------------
% Abstract in Hungarian
%----------------------------------------------------------------------------
\chapter*{Kivonat}\addcontentsline{toc}{chapter}{Kivonat}

A monitorozással történő hibadetektálás kiemelt fontosságú egy kritikus rendszer működtetésében és karbantartásában.
A monitorozás sok hibát fel tud deríteni, amiket a tesztek nem feltétlenül tudnak kideríteni.

A diplomaterv feladat célja az volt, hogy a korábban elkészített monitor komponens generátort kiegészítsem úgy, hogy időzített üzenet szekvencia specifikáció alapján is képes legyen monitor komponenseket generálni.
Ilyen követelményeket egyszerűen specifikálhatunk TPSC (Timed Property Sequence Chart) diagramokkal. A korrábban elkészített Xtext alapú PSC nyelvet kiegészítettem a TPSC tulajdonságaival.

A monitor generálás következő lépése, hogy a TPSC diagramokból időzített automatákat generálunk (Timed Automaton).
A TA fogja megadni, hogy a megfigyelt kommunikáció helyes viselkedést jelent-e.
A szakdolgozatomban készített automata generátort kibővítettem és most már képes a minta alapú módszert használva TA automatákat generálni a TPSC diagramjainkból.
A generátor előállítja az automatát.

A szöveges scenario leírásból generált automata alapján legenerálható a monitor forráskódja.
A monitor forráskód generátor előállítja a monitor \textit{Java} implementációját, ami képes egy rendszer monitorozására adott követelmény alapján.

A diplomaterv további feladatai közzé tartozik a TPSC scenario-k vizualizációja, a generált forráskódok tesztelése és a generált monitor komponens illesztése a Gamma keretrendszerrel.
A cél az, hogy elosztott komponens alapú rendszerek szimulációja közben monitorozható legyen a TPSC üzenet szekvencia specifikáció teljesülése illetve az ebben rögzített tulajdonságok megsértése.
Továbbá a monitorozás működését is demonstrálni kell.

Az időzítési feltételeket tartalmazó üzeneteket tudja értelmezni és feldolgozni a monitor, valamint a teszt keretrendszer első verziója is elkészült.

\vfill
\selectenglish


%----------------------------------------------------------------------------
% Abstract in English
%----------------------------------------------------------------------------
\chapter*{Abstract}\addcontentsline{toc}{chapter}{Abstract}

% TODO: összefoglaló angol verziójának megírása
This document is a \LaTeX-based skeleton for BSc/MSc~theses of students at the Electrical Engineering and Informatics Faculty, Budapest University of Technology and Economics. The usage of this skeleton is optional. It has been tested with the \emph{TeXLive} \TeX~implementation, and it requires the PDF-\LaTeX~compiler.


\vfill
\selectthesislanguage

\newcounter{romanPage}
\setcounter{romanPage}{\value{page}}
\stepcounter{romanPage}