%----------------------------------------------------------------------------
\chapter{TPSC – Timed Property Sequence Charts}
%----------------------------------------------------------------------------
A TPSC[2] a PSC-nek egy kiterjesztése. A PSC üzenetekre időzítési feltételeket specifikálhatunk.

% TODO: kép beíllesztése
A TPSC óraváltozókat (x, y) használ az időzítéshez. Ezekre meg lehet adni feltételeket, valamint az óraváltozót lehet nullázni. A nullázással adott eseménytől (pl. üzenet vételétől) kezdve lehet időzítést indítani, majd rákövetkező események időbeliségét ellenőrizni.

A 4. ábrán látható, hogy például az e: a sima üzenet e: a; x < t, y := 0 üzenetre bővül. Elvárjuk, hogy az a üzenet t idő előtt történjen meg és egy y óraváltozót nullázunk. Az e: a üzenet egy sima üzenet, szóval ha nem történik meg a specifikált idő intervallumban az nem jelent hibát. Viszont r: a üzenetnél már elvárt, hogy t időn belül megtörténjen. f: a üzenet esetében viszont akkor jelez hibát a monitor, ha üzenet megtörtént t időn belül.

Egy megkötésre is meg lehet adni időzítési feltételt. Így megadhatjuk, hogy mennyi ideig nem szabad jönnie a megkötésben szereplő nem kivánt üzenetek egyikének. Ha a feltételben megadott idő után történik akkor az nem jelent hibát a monitor szempontjából.