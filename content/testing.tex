%----------------------------------------------------------------------------
\chapter{Tesztelési terv}
%----------------------------------------------------------------------------

%----------------------------------------------------------------------------
\section{Időzített automata generátor tesztelése}
%----------------------------------------------------------------------------

A generált időzített automaták forráskodját unit tesztek segítségével szeretnénk tesztelni.
Az Xtext keretrendszer által nyujtott eszközök erre célra jól alkalmazhatok.

\begin{figure}[!ht]
    \centering
    \includegraphics[width=150mm, height=9cm, keepaspectratio]{figures/unit_test_flow.png}
    \caption{Unit tesztelés folyamatábrája.}
\end{figure}

A 9.1. ábrán látható a unit tesztelés tervének folyamatábrája.
A teszteléssel az a célunk, hogy minél nagyobb magabiztosággal biztosítjuk a generátor helyes működését.
Tesztelési kategoriák:

\begin{itemize}
    \item Üzenet típusok egyenkénti tesztelése
    \item Üzenet típusokból összeállított kombinációk tesztelése
    \item Scenario operátorok tesztelése
\end{itemize}

Az üzenet típusok egyenkénti tesztelésénél az összes üzenet típust le szeretnénk fedni.
Az üzenet kombinációkból a főbb eseteket szeretnénk tesztelni.
Például sima üzenet után required üzenet és ezek váltakozása a fail üzeneteket is bele értve.
A scenario operátoroknál azt fontos tesztelni, hogy a különálló üzenetekhez tartozó automata minták jól illeszkedjenek a operátorral elátott üzenetek automatájával.

Egy unit teszt akkor sikeres ha a generált automata forráskódja megegyezik az elvárt automata forráskodjával.

%----------------------------------------------------------------------------
\clearpage\section{Monitor forráskód generátor tesztelése }
%----------------------------------------------------------------------------

A generált monitor forráskodját integrációs tesztek segítségével szeretnénk tesztelni.
Az Xtext keretrendszer a specifikált dsl nyelvhez generál egy maven plugin-t.
Ezt a plugin-t betölthetjük egy egyszerű maven projektbe és használhatjuk is az elkészített dsl nyelvünket.

\begin{figure}[!ht]
    \centering
    \includegraphics[width=150mm, height=9cm, keepaspectratio]{figures/integration_test_flow.png}
    \caption{Integrációs tesztelés folyamatábrája.}
\end{figure}

A 9.2. ábrán megtekinthető az integrációs tesztelés tervének folyamatábrája.
A következők tesztelési kategoriáink:

\begin{itemize}
    \item Scenario operátorok tesztelése
    \item Üzenet paraméterek tesztelése
    \item Időzítések tesztelése
\end{itemize}

A scenario operátorok tesztelésénél az a célunk, hogy a monitor a követelmény különböző ágait figyelembe véve helyes kimenetet adjon.
Például loop operátor esetén a minimum, köztes és maximum üzenet szekvencia ismétléseknél is helyes legyen a monitor kimenete.
Ha a maximumnál többször szerepel az üzenet szekvencia akkor hibát kell, hogy jelezen.
Üzenet paraméterek tesztelése esetén azt szeretnénk vizsgálni, hogy a monitor helyesen értelmezi e az üzenet paramétereket.
Az időzítések tesztelésénél az a fontos, hogy a monitor képes-e az óraváltozók alapján az időzitési feltételeket kiértékelni.
Például ha az üzenet a feltétel alapján időben érkezik meg akkor helyes kimenetet adjon, vissza ha feltétel szerint később érkezik meg akkor a monitornak hibát kell jeleznie.

Egy integrációs teszt akkor sikeres ha a monitor kimenete megegyezik az elvárt kimenettel.

%----------------------------------------------------------------------------
\clearpage
%----------------------------------------------------------------------------

Az Xtext keretrendszer a definiált DSL nyelvünkhöz generál egy Maven projekt architektúrát.
A nyelvünk így elérhető maven plugin formájában is, amit felhasználhatunk az integrációs teszteinkhez.
Elég csupán egy maven projektet felkonfigurálni a saját dsl plugin-ünkkel és elkészítethetjük a saját tesztelési keretrendszerünket.
A 9.3. ábrán és a 9.1. kódrészléten látható egy ilyen integrációs teszthez tartozó maven projekt felépítése és a hozzá tartozó teszteset.

\begin{figure}[!ht]
    \centering
    \includegraphics[width=150mm, height=9cm, keepaspectratio]{figures/integration_test_structure.png}
    \caption{Példa integrációs teszt projekt struktúrája.}
\end{figure}

\begin{lstlisting}[language=java, frame=single, float=ht!, caption={Integrációs teszteset.},captionpos=b]
package hu.bme.mit.dipterv.text.example;

import org.junit.jupiter.api.Test;
import org.junit.jupiter.api.Assertions;

import generated.Specification;
import generated.IClock;
import generated.IMonitor;
import generated.Monitor;
import generated.Clock;

public class MonitorPassingTest {

	@Test
	public void testMonitorPassing() {
		Specification specification = new Specification();
		specification.listAutomatas();
		IClock clock = new Clock();
		IMonitor monitor = new Monitor(specification.getAutomata().get(0), clock);

		Server server = new Server(monitor);
		Computer computer = new Computer(server, monitor);
		Assertions.assertTrue(monitor.goodStateReached());
	}
}
\end{lstlisting}

\begin{lstlisting}[language=java, frame=single, float=ht!, caption={Integrációs teszteset eredménye.},captionpos=b]
-------------------------------------------------------
T E S T S
-------------------------------------------------------
Running hu.bme.mit.dipterv.text.example.MonitorPassingTest
q0 NORMAL
q1 NORMAL
q2 ACCEPT
q3 NORMAL
q4 NORMAL
q5 FINAL
!(computer.checkEmail().computer) q0->q0
computer.checkEmail().computer q0->q1
!(computer.sendUnsentEmail().server) q1->q1
!(computer.sendUnsentEmail().server) q1->q2
computer.sendUnsentEmail().server q1->q3
!(computer.logout().server) & !(computer.newEmail().server) q3->q3
computer.newEmail().server q3->q4
!(computer.downloadEmail().server) q4->q4
computer.downloadEmail().server q4->q5
Received Message: computer.checkEmail().computer
Transition: !(computer.checkEmail().computer)
Transition: computer.checkEmail().computer
transition triggered: computer.checkEmail().computer
q1

Received Message: computer.sendUnsentEmail().server
Transition: !(computer.sendUnsentEmail().server)
Transition: !(computer.sendUnsentEmail().server)
Transition: computer.sendUnsentEmail().server
transition triggered: computer.sendUnsentEmail().server
q3

Received Message: computer.newEmail().server
Transition: !(computer.logout().server) & !(computer.newEmail().server)
Transition: computer.newEmail().server
transition triggered: computer.newEmail().server
q4

Received Message: computer.downloadEmail().server
Transition: !(computer.downloadEmail().server)
Transition: computer.downloadEmail().server
transition triggered: computer.downloadEmail().server
q5

Tests run: 1, Failures: 0, Errors: 0, Skipped: 0, Time elapsed: 0.025 sec

Results :

Tests run: 1, Failures: 0, Errors: 0, Skipped: 0
\end{lstlisting}

A 9.2. kódrészlet a Maven teszt kimenetét tartalmazza.

Ezt a projekt struktúrát felhasználva a teszteink köré tudunk egy Maven alapú Continuous Integration (CI) állítani.

A függelékben található egy példa unit teszt eset az automata generátor tesztelésére.