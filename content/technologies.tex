\clearpage\section{Használt technológiák}
\subsection{Eclipse}

Az \textit{Eclipse} egy nyílt forráskódú, platformfüggetlen keretrendszer.
Első sorban fejlesztői környezetként használják a fejlesztők.
A keretrendszert tovább lehet bővíteni mindenféle plugin telepítésével, így például modellezésre is alkalmas lehet.
A diplomaterv során használt Eclipse plugin-ek:

\begin{itemize}
    \item Xtext
    \item Xtend
    \item Sirius
\end{itemize}

\subsection{Xtext}

Az \textit{Xtext} \textit{Eclipse} plugin-el programozási és domain specifikus nyelveket lehet fejleszteni.
A nyelvünk elemeit és szabályait egy nyelvtan segítségével definiálhatjuk.
Az \textit{Xtext} keretrendszer több eszközt nyújt a nyelvünkhöz.
Például egy parser-t, egy fordítót és egy szerkesztőt.
A plugin még egy \textit{Xtend} alapú kódgenerátort is generál a nyelvünkhöz, amivel a nyelvünkhöz tetszőleges kódot tudunk generálni.

\subsection{Xtend}

Az \textit{Xtend} egy magasszintű programozási nyelv.
A \textit{Java Virtual Machine} plaformot használja.
Szintaktikailag és szemantikailag nagyon hasonlít a \textit{Java} nyelvhez, mondhatni a Java kibővítése.
Az \textit{Xtend} osztályokból \textit{Java} osztályok készülnek.

\subsection{Sirius}

A \textit{Sirius} eszköz segítségével létrehozhatunk saját grafikus modellező alkalmazásainkat.
Egy szerkesztő környezetet alkothatunk, amivel a modellünk elemeit hozhatjuk létre vagy szerkeszthetjük grafikusan.
A plugin az \textit{Entity Modelling Framework} keretrendszer használja a modellek feldolgozásához.