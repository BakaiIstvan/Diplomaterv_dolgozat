\clearpage\section{Használt technológiák}
\subsection{Eclipse}

Az textit{Eclipse} egy nyílt forráskódú, platformfüggetlen keretrendszer.
Első sorban fejlesztői környezetként használják a fejlesztők.
A keretrendszert tovább lehet bővíteni mindenféle plugin telepítésével, így például modellezésre is alkalmas lehet.
A diplomaterv soránt használt Eclipse plugin-ek:

\begin{itemize}
    \item Xtext
    \item Xtend
    \item Sirius
\end{itemize}

\subsection{Xtext}

Az textit{Xtext} Eclipse plugin-el programozási és domain specifikus nyelveket lehet fejleszteni.
A nyelvünk elemeit és szabályait egy nyelvtan segítségével definiálhatjuk.
Az textit{Xtext} keretrendszer több eszközt nyújt a nyelvünkhöz.
Például egy parser-t, egy fordítót és egy szerkesztőt.
A plugin még egy Xtend alapú kódgenerátort is generál a nyelvünkhöz, amivel a nyelvünkhöz tetszőleges kódot tudunk generálni.

\subsection{Xtend}

Az Xtend egy 

\subsection{Sirius}