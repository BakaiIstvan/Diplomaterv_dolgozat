%----------------------------------------------------------------------------
\chapter{Források}
%----------------------------------------------------------------------------
\begin{itemize}
    \item [1] M.Autili – P. Inverardi – P.Pelliccione, Graphical scenarios for specifying temporal properties: an automated approach in Automated Software Engineering 14(3):293-340, September
2007, \url{https://link.springer.com/article/10.1007%2Fs10515-007-0012-6}
    \item [2] Pengcheng Zhang - Hareton Leung, Web services property sequence chart monitor: A tool chain for monitoring BPEL – based web service composition with scenario-based specifications in IET Software 7(4):222-248, August
2013
    \item [3] Xtext, \url{https://www.eclipse.org/Xtext/}
    \item [4] Xtend, \url{https://www.eclipse.org/xtend/}
    \item [5] J. Ouaknine and J. Worrell, “On Metric Temporal Logic and Faulty Turing Machines,” Springer-Verlag, FOSSACS, vol. LNCS 3921, pp. 217-230, 2006.
    \item [6] ITU-T Recommendation Z. 120.: Message sequence charts. ITU Telecom. Standardisation Sector (1999)
    \item [7] M. Leucker, „Teaching Runtime Verification”, 2011 - Springer
    \item [8] Object Management Group (OMG): UML: superstructure version 2.0 (2004)
\end{itemize}